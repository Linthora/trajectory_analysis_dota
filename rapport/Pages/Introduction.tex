\chapter{Introduction}

%Projet scientifique sur le thème des sciences des données visant à réaliser une analyse de trajectoires (ici appliqué sur des données du jeu Dota 2). Pour y arriver, nous avons procédé avec les méthodes suivantes: compression s'appuyant sur le principe MDL, discrétisation par clustering (K-means, K-medoids ou Propagation d'affinités) et extraction séquentielle des motifs fréquents.

Depuis de nombreuses années, l'analyse de données est devenue un domaine clé de notre évolution. En effet, les sciences des données permettent de traiter de plus en plus aisément de grandes quantités de données et d'en extraire des connaissances pertinentes.

\bigskip

Dans notre projet, nous nous concentrons sur un domaine spécifique de la science des données~: l'analyse de trajectoires. Tout comme la génétique a utilisé la drosophile comme modèle pour comprendre les principes fondamentaux de l'hérédité, nous utiliserons les jeux vidéo pour explorer les principes sous-jacents de l'analyse de trajectoires, ici, il s'agira du jeu Dota 2.

\vspace{0.3cm}

Les jeux vidéo offrent un environnement idéal pour l'analyse de trajectoires, car ils fournissent des données riches et précises sur les mouvements des joueurs, leurs interactions avec l'environnement et les stratégies de jeu. En utilisant ces données, nous pouvons explorer les motifs de jeu, identifier les comportements des joueurs et améliorer à terme les performances tactiques. 

\vspace{0.3cm}

L'identification des motifs de jeu requiert de discrétiser les trajectoires afin d'obtenir pour chaque joueur des suites de symboles plutôt que des coordonnées. Ce passage au symbolique permet de comparer les comportements collectifs grâce à des méthodes de découverte de motifs fréquents. Dans cette étude, nous utilisons donc des méthodes de compression, de clustering et d'extraction séquentielle pour analyser les trajectoires de Dota2.

\vspace{0.3cm}

Nos objectifs sont :
\begin{enumerate}
    \item Illustrer les différences entre les deux algorithmes de clustering que nous utiliserons, à savoir K-means et Affinity Propagations.
    \item Discuter de l'intérêt des régularités découvertes dans nos données.
\end{enumerate}

\vspace{0.3cm}

La problématique est de réussir à mettre en place cette chaîne de traitements successifs des données et de paramétrer chaque étape pour obtenir des résultats pertinents.

\vspace{0.3cm}

Dans cette optique, nous allons donc commencer par expliquer l'organisation de ces traitements, puis nous développerons sur le fonctionnement de chacune de ces étapes, pour finalement regarder et discuter les résultats obtenus à chaque étape.


%Nous présentons les résultats de notre analyse et discutons de leur pertinence pour améliorer les performances tactiques dans les jeux vidéo. Cette étude montre l'importance de la science des données dans l'analyse de trajectoires, ainsi que le potentiel des jeux vidéo comme outil de recherche pour explorer les principes fondamentaux de l'analyse de données.


%Question scientifique : illustrer les différences entre les algorithmes des k-moyennes et propagation d'affinité, puis discuter l'intérêt des régularités découvertes.

%Projet scientifique sur le thème des sciences des données visant à réaliser une analyse de trajectoires (ici appliqué sur des données du jeu Dota 2). Pour y arriver, nous avons procédé avec les méthodes suivantes: compression s'appuyant sur le principe MDL, discrétisation par clustering (K-means, K-medoids ou Propagation d'affinités) et extraction séquentielle des motifs fréquents.

% \section{Objectif }

% \section{Problématique}

% \section{Plan d'attaque}